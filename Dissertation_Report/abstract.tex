\begin{abstract}
At HKU, the Programming Languages Group is developing a research middleware compiler (F2J), which converts System F-based languages to Java with the imperative functional object encoding.  In current compiler, recursive function calls often cause stack overflow and this is because the number of stack frames generated in every iteration grow to be very significant. When the number of iterations is too big, without appropriate control of the stack space, then it would very much likely to explode and thus causing stack overflow.
Here, we present a solution to this problem, which is Continuation-Passing Style transformation (CPS). This transformation converts every function call in the program to tail call form, which asymptotically reduces the stack space requirements from linear to constant. Besides, our CPS algorithm operates in one pass and is both compositional and first-order. It not only makes the control flow in the program explicit but also contains optimizations for reducing the number of administrative redexes generated in the translation process. 
\end{abstract}
